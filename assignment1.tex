\documentclass[a4paper]{article}

\usepackage{enumerate}
\usepackage{amsmath}
\usepackage{hyperref}
\usepackage{xfrac}

\title{Assignment 1}
\author{Jacco Spoelder (s1348493) \and Xeryus Stokkel (s2332795)}

\begin{document}

\maketitle

\section{Top 500 list}

\section{Speedup}
\begin{enumerate}[(a)]
	\item Speed up is the decrease in time required to solve a problem when using multiple parallel processors. It is defined as
		\begin{equation}
			S(p) = \frac{t_1}{t_p} \label{eqn:base}
		\end{equation}
		where $t_1$ is the time required to solve the problem on a single processor and $t_p$ is the time required to solve the problem on $p$ processors.
	\item Any program can be split up in a part that can be parallised ($t_\text{par}(1)$) and a part which can not ($t_\text{seq}(1)$). We can then define $t_1$ and $t_p$ as follows:
		\begin{align*}
			t_1 &= t_\text{seq}(1) + t_\text{par}(1) \\
			t_p &= t_\text{seq}(1) + \frac{t_\text{par}(1)}{p}
		\end{align*}
		This can be combined with the \autoref{eqn:base} to form:
		\begin{align}
			S(p) &= \frac{t_\text{seq}(1) + t_\text{par}(1)}{t_\text{seq}(1) + \frac{t_\text{par}(1)}{p}} \nonumber \\
			S(p) &= \frac{\frac{t_\text{seq}(1)}{t_\text{par}(1)} + 1}{\frac{t_\text{seq}(1)}{t_\text{par}(1)} + \frac{1}{p}} \label{eqn:step2}
		\end{align}
		\autoref{eqn:step2} contains the ratio $\displaystyle \frac{t_\text{seq}(1)}{t_\text{par}(1)}$, of which we can take the inverse to get the desired ratio $\displaystyle x = \frac{t_\text{par}(1)}{t_\text{seq}(1)}$ so we can rewrite it as follows:
		\begin{align*}
			S(p) &= \frac{x^{-1} + 1}{x^{-1} + p^{-1}}
		\end{align*}
	\item To get the value of the ratio $x$ we need to solve the following:
		\begin{align*}
			200 &= \frac{x^{-1} + 1}{x^{-1} + 300^{-1}} \\
			200 (x^{-1} + 300^{-1}) &= x^{-1} + 1 \\
			200 x^{-1} + \sfrac{2}{3} &= x^{-1} + 1 \\
			199 x^{-1} &= \sfrac{1}{3} \\
			x^{-1} &\approx \sfrac{1}{600} \\
			x &\approx 600
		\end{align*}
	\item Since $S(p) \approx x$ for $p \gg x$ the maximum speed-up is 600. For example we could use $10^9$ processors: $\displaystyle S(10^9) = \frac{600^{-1} + 1}{600^{-1} + (10^9)^{-1}} \approx 601$, as we can see the speed-up with such a large number of processors is still only 600.
	\item To achieve half of the maximum speed up we would need to solve $S(p) = 300$:
		\begin{align*}
			300 &= \frac{600^{-1} + 1}{600^{-1} + p^{-1}} \\
			300 (600^{-1} + p^{-1}) &= 600^{-1} + 1 \\
			\frac{300}{p} + \frac{1}{2} &= 600^{-1} + 1 \\
			\frac{300}{p} &= \frac{1}{2} + 600^{-1} \\
			p &\approx 600
		\end{align*}
		So about 600 processors are needed to get half of the maximum speed-up.
\end{enumerate}

\section{Amdahl's and Gustafson's law}

\end{document}