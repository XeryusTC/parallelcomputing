\documentclass[a4paper]{article}

\usepackage{enumerate}
\usepackage{amsmath}

\title{Assignment 1}
\author{Jacco Spoelder (s1348493) \and Xeryus Stokkel (s2332795)}

\begin{document}

\maketitle

\section{Top 500 list}
The end of (one of) our student numbers is 93. At the 93th positon in the current high performance computer top 500 list, as of November 2014, is the SANAM Supercomputer, located in Saudi Arabia. It appeared first in the top 500 list at the end of 2012. It was built by and hosted in the King Abdulaziz City for Science and Technology(KACST) in cooperation with Frankfurt Institute for Advanced Studies(FIAS). At the time it was completed, it ranked the worlds second most efficient supercomputer, based on the performance/energy use ratio. This efficiency is expressed as the computing power in floating point operation per second(FLOPS) per watt of energy used. The SANAM supercomputer is able to do 2,351.10 mega FLOPS per watt. Nowadays it is ranked number 23. The most efficient supercomputer at this moment, the GSI Helmholtz Center based L-CSC, performs 5,271.81 mega FLOPS per watt, more than twice as efficient. At the time of completion the SANAM placed at position 52 in the top 500 supercomputer list, while, as said, it now has position 93. 
\\
The high efficiency and total computing power of the SANAM system is acquired through the use of many general purpose graphic processing units(GPGPU) instead of only central processing units(CPU).
\section{Speedup}
\begin{enumerate}[\textbf{(a)}]
	\item Speed up is the decrease in time required to solve a problem when using multiple parallel processors. It is defined as $S(p) = \frac{T_1}{T_p}$ where $T_1$ is the time required to solve the problem on a single processor and $T_p$ is the time required to solve the problem on $p$ processors.
	\item
		\begin{align}
			T_1 &= T_\text{seq} + T_{par}
		\end{align}
\end{enumerate}

\section{Amdahl's and Gustafson's law}

\end{document}